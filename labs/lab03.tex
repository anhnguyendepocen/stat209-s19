\documentclass{tufte-handout}

\usepackage{amssymb,amsmath}
% \usepackage{mathspec}
\usepackage{graphicx,grffile}
\usepackage{longtable}
\usepackage{booktabs}

\newtheorem{mydef}{Definition}
\newtheorem{thm}{Theorem}

\providecommand{\tightlist}{%
  \setlength{\itemsep}{0pt}\setlength{\parskip}{0pt}}

\setlength{\parindent}{0em}
\setlength{\parskip}{12pt}

\begin{document}

\justify

{\LARGE Lab 03}

\vspace*{18pt}

Download the \texttt{lab03.Rmd} file and open it using RStudio. Then, use the
R programming language to help you answer the questions below. \textbf{Don't
forget to fill out the worksheet form before the next class!}

\vspace*{12pt}

\textbf{1.} Consider the experiment you devised and collected data for in our
last class. Describe the value of following elements in your study:
(i) null hypothesis, (ii) alternative hypothesis, (iii) test statistic,
(iv) sampling distribution, (v) p-value, and (vi) the statistical significance
of the outcome.

\textbf{2.} Imagine a new \$2 U.S. coin. Denote the value $p$ to be the
probability that the coin comes up heads when flipped. Assume that you run an
experiment $300$ times in which the coin comes up heads only $125$ times.
Describe the value of following elements from this study (use the code in the
lab to compute elements iv and v): (i) null hypothesis, (ii) alternative
hypothesis, (iii) test statistic, (iv) sampling distribution, (v) p-value,
and (vi) the statistical significance of the outcome.

\textbf{3.} Find the Wikipedia page on the Null Hypothesis. Read the page up
to and including the Example section. Take note of any new points,
inconsistencies, or points of confusion with our description of the null
hypothesis today.

\textbf{4.} Consider an experiment to test whether students who wear glasses
are taller than students who do not wear glasses. Describe the
(i) null hypothesis, (ii) alternative hypothesis, and (iii) a reasonable
test statistic. Sketch what you think the sampling distribution of the test
statistic might look like.

\textbf{5.} I cannot stress enough how important it is to internalize and
understand the concepts discussed today. They are central to our study of
statistics during the entire semester. Find a partner and re-explain the
five concepts on a piece of paper or a white board without any notes.

\end{document}

