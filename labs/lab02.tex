\documentclass{tufte-handout}

\usepackage{amssymb,amsmath}
% \usepackage{mathspec}
\usepackage{graphicx,grffile}
\usepackage{longtable}
\usepackage{booktabs}

\newtheorem{mydef}{Definition}
\newtheorem{thm}{Theorem}

\providecommand{\tightlist}{%
  \setlength{\itemsep}{0pt}\setlength{\parskip}{0pt}}

\setlength{\parindent}{0em}
\setlength{\parskip}{12pt}

\begin{document}

\justify

{\LARGE Lab 01}

\vspace*{18pt}

Download the \texttt{lab01.Rmd} file and open it using RStudio. Then, use the
R programming language to help you answer the questions below. \textbf{Don't
forget to fill out the worksheet form before the next class!}

\vspace*{12pt}

\textbf{1.} You can use R as an overpowered calculator. Compute the sum of
$14^2 + 6^4$ by typing the expression into the grey box on line 9 of the
Rmd file. Note that to use powers you can type something like \verb|2^2|
(the shifted version of the 6 key on a U.S. keyboard). To run the code, click
on the green play button at the end of line 8.

\bigskip

\textbf{2.} You can also save the result of a computation by using the arrow
symbol \texttt{<-} and saving the result as a \textit{variable}. On line 15,
I have already written the code to assign the result of \texttt{2 + 2} to the
variable \texttt{myvar}. Run the code and notice that result shows up in the
\textit{Environment pane} in the upper right hand corner of the screen.

\bigskip

\textbf{3.} R also allows us to simulate random values. Run the code I wrote
on line 21 to generate 10 random numbers between $0$ and $1$, save it as the
variable \texttt{rvals}, and then print out the results. Run the code a couple
of times to see that the output changes each time. Can you figure out what the
numbers in square brackets (like \texttt{[1]} and \texttt{[9]}) mean?

\bigskip

\textbf{4.} The basic version of R that you have downloaded can do a lot, but
the real power of the programming language comes from additional components
called \textit{packages}. The code block in question 4 installs four packages
that we will need this semester. It will also install other packages that are
needed by these three (in all, it's about 60 and may take a few minutes).

\bigskip

\textbf{5.} Once you have downloaded the packages (question 4), you need to
also load them using the \texttt{library()} function. Run the code here to
load the \textbf{dplyr} package. It may produce some warnings in red, but
unless it actually uses the word ``Error'' you should be fine.

\bigskip

\textbf{6.} There are two basic types of R objects that we will work with
this semester. The first is called a \textit{vector}, consisting of one or
more ordered values. You already saw an example of this with the object we
created called \texttt{rvals}. The second object type is called a \textit{data
frame} or \textit{tibble}. A data frame contains a grid of values, similar to
an excel sheet. The code in this question creates a random (entirely
meaningless) example of a data frame and saves it as the object
\texttt{dframe}. Notice that after running the code, the object shows up in
the \textit{Environment pane} in the upper right hand corner of the screen.
Click on the data frame in the \textit{Environment pane} to see a tabular
version of the data.



\end{document}

