\documentclass[12pt]{article}

\usepackage{fontspec}
\usepackage{geometry}
\usepackage{lastpage}
\usepackage{fancyhdr}
\usepackage{hyperref}

\geometry{top=1in, bottom=1in, left=1in, right=1in, marginparsep=4pt, marginparwidth=1in}

\renewcommand{\headrulewidth}{0pt}
\pagestyle{fancyplain}
\fancyhf{}
\lfoot{As of 2018-02-14}
\rfoot{page \thepage\ of \pageref{LastPage}}

\setlength{\parindent}{0pt}
\setlength{\parskip}{0pt}

% \setromanfont [Ligatures={Common}, Numbers={OldStyle}, Variant=01,
%  BoldFont={LinLibertine_RB.otf},
%  ItalicFont={LinLibertine_RI.otf},
%  BoldItalicFont={LinLibertine_RBI.otf}
%  ]{LinLibertine_R.otf}

\usepackage{tikz}
\def\checkmark{\tikz\fill[scale=0.4](0,.35) -- (.25,0) -- (1,.7) -- (.25,.15) -- cycle;}

\usepackage{xunicode}
\defaultfontfeatures{Mapping=tex-text}

\setromanfont{YaleNew}

\begin{document}

\begin{center}
{\bf MATH/STAT 209: Introduction to Statistical Modelling, Spring 2019} \\
Tuesday, Thursday 12:00-13:15 \quad MRC LL1\\
Tuesday, Thursday 13:30-14:45 \quad MRC LL1
\end{center}

\bigskip

\noindent
\begin{tabular}{ l l }
{\bf Instructor:} &  {\bf Taylor Arnold} \\
E-mail: & \href{mailto:tarnold2@richmond.edu}{tarnold2@richmond.edu} \\
Office: & Jepson Hall, Rm 218 \\
Office hours: & Monday 15:00-16:00 and 17:30-18:30
\end{tabular}

\vspace{0.5cm}

\textbf{Description:} \vspace{6pt}

This course broadly covers the entire processing of collecting, visualizing,
and modeling data. It has a MATH designation but is not a \textit{mathematics}
course. The focus is on applied statistics and data analysis rather than a
detailed study of symbolic mathematics. By the end of the semester you will
feel confident collecting, analyzing, and writing about datasets from a
variety of fields. You will be able to use these skills to address data-driven
problems in a wide range of application domains.

\bigskip

\textbf{Computing:} \vspace{6pt}

To facilitate your ability to actually \textit{do} statistics, most class
meetings will involve some form of computing. No prior programming experience
is assumed or required.

\medskip

We will use the \textbf{R} programming environment throughout the
semester. It is freely available for all major operating systems and
is pre-installed on many campus computers. You can download it and
all supporting files for your own machine via these links:
\begin{center}
\url{https://cran.r-project.org/} \\
\url{https://www.rstudio.com/}
\end{center}
You are required to bring a laptop to each class meeting with R installed
and running. This requires that you have a computer with an up-to-date
version of macOS, Windows, or Linux (iPads and Chromebooks will not suffice).
If this is not possible, or becomes a problem during the semester, it is your
responsibility to inform me as soon as possible so that we can find an
alternative solution.

\bigskip

\textbf{Course Website:} \vspace{6pt}

All of the materials and assignments for the course will be posted
on the class website:
\begin{quote}
\url{https://statsmaths.github.io/stat209-s19}
\end{quote}
The website contains notes, assignment details, and supplemental materials.
At the end of the semester, this version of the course will be archived and
available for your reference.

\vspace{0.4cm}

\textbf{Labs:} \vspace{6pt}

During most class meetings, you will work on a series of assignments I refer
to as `Labs'. These may be a paper handout with questions or a code file that
requires that you fill in answers digitally. In order to succeed in the course
you should complete these prior to the next class meeting. Rather than
formally handing them in, you must instead fill out an online questionnaire
at some point prior to the next class meeting. I will not accept late
submissions. The questionnaire can be found through a link on the course
website. You have a grace period of forgetting to hand in \textbf{two}
questionnaires. Beyond these you will lose one point on your final grade for
every missing questionnaire.

\vspace{0.4cm}

\textbf{Exams:} \vspace{6pt}

We will have three exams given during the semester. Exams will focus on
the material in each section of the course, but due to the cumulative nature
of the material each requires understanding previous sections. There may be
in-class and take-home components of each exam. The in-class portion will take
place on the following days:

\begin{itemize}\setlength\itemsep{0em}
\item 2019-02-12 (Tue)
\item 2019-03-21 (Thur)
\item 2019-04-18 (Thur)
\end{itemize}

Any take-home component will be due the class \textit{after} the day on which
the in-class exam is assigned.

\vspace{0.4cm}

\textbf{Final Project:} \vspace{6pt}

In lieu of a final exam, the course concludes with a final project.
The project is due on the second to last class of the semester so that we can
accommodate in-class presentations during the final week of the semester.
More details on the project will be given prior to Spring Break.

\vspace{0.4cm}

\textbf{Final Grades:} \vspace{6pt}

You will receive a numeric score from 0-100 for each exams and the final
project. Your final numeric grade is determined by taking the average score
of your three best grades, rounded to the nearest integer. Finally, subtract
one point for every missing lab/worksheet for which you failed to turn a form
in (beyond the grace window of two missing forms).

\medskip

The mapping from numeric grades to letter grades is given as follows:
\begin{itemize}\setlength\itemsep{0em}
\item[] \textbf{A} $\Rightarrow$ 90 to 100
\item[] \textbf{B} $\Rightarrow$ 80 to 89
\item[] \textbf{C} $\Rightarrow$ 70 to 79
\end{itemize}
I may assign pluses and minuses as needed. When appropriate, I may also modify
these cut-off scores to make them more generous (but will not make them more
strict).

\vspace{0.4cm}

\textbf{Attendance:} \vspace{6pt}

There is no formal attendance policy for this course. However, if you miss a
class it is your responsibility to catch up with the material. E-mail and
office hours are not a replacement for attendance. When you are present,
I expect you to arrive on time, engage with the material, and give us your
full attention.

\vspace{0.4cm}

\textbf{Class Policies:} \vspace{6pt}

The following class policies address some of the most common
questions and concerns that students have. If anything is
unclear, please feel free to contact me for clarification at
any point in the semester.

\begin{itemize}\setlength\itemsep{0em}
\item \textbf{Academic honesty:} Cheating and plagiarism are grave scholarly
offenses and potential grounds
for expulsion; they are also a major barrier to your intellectual development.
You are expected to familiarize yourself with the entirety of the
University of Richmond’s Honor Code. If you are confused or unsure about
appropriate citation protocol or any other aspect of the Honor code,
please consult me before turning in an assignment.
\item \textbf{Special approval:} If you have special approval forms for extra
time on exams or any other circumstances I should know about, please speak
with me as early as possible so that we can best accommodate your needs.
\item \textbf{Late work:} You are expected to submit all work on-time. The
final project will be accepted after the due date with a 10-point deduction
for each 24 hour period it is late (rounded up).
\item \textbf{Make-up exams:} There are no make-up exams. If you fail to
attend an exam without a valid excuse (given to me by email within 24-hours
of the exam) you will receive a score of zero. In the event of a valid reason
for missing the exam, the missing score will be filled in with the median
grade from the remaining elements.
\item \textbf{Class conduct:} During class I expect you to refrain from checking
email, being on phones, or working on assignments for other classes.
\item \textbf{Office hours}: If you would like to meet during my office hours,
please just come by. No need to schedule an appointment. If you find me in
my office at other times of the week, I am usually glad to meet then as well.
Finally, I am also happy to make appointments outside of my normal
office hours. These appointments are meant for discussing
longer issues that are not appropriate for regular office hours (i.e., asking
for recommendation letters or discussing an extended absence) or for students
who cannot make my normal office hours. Please note that appointments should
be booked at least 24 hours ahead of time.
\item \textbf{Email:} I will also answer questions by email (it can, in fact,
be much faster than scheduling an appointment for small issues). During the
week, I aim to respond within 24 hours, with emails sent over the weekend
responded to by Monday morning. If your question involves code, please attach
your current lab or report as that will expedite my answering your question(s).
\end{itemize}

\bigskip

\textbf{Notice:} \vspace{6pt}

I reserve the right to modify this syllabus, with advanced warning, throughout
the semester. If necessary, I will email the class list and post an updated
version of the document on the course website.


\end{document}





